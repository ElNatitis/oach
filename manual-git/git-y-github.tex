\documentclass[10pt,a4paper]{book}
\usepackage[utf8]{inputenc}
\usepackage{amsmath}
\usepackage{amsfonts}
\usepackage{amssymb}
\usepackage{graphicx}
\usepackage[T1]{fontenc}    % Mejor renderizado de fuentes
\usepackage{upquote}        % Corrige la visualización de `~` y `` ` ``
\usepackage{listings}       % Para mostrar código en un entorno controlado
\usepackage{xcolor}         % Para colores (opcional)
\usepackage{tabularx} 		  % Para tablas ajustables
\usepackage{makecell} 		  % Para saltos de línea en celdas
\usepackage[spanish]{babel} % Configura el idioma a español
\usepackage{hyperref}       % Para hipervínculos en el PDF

% Configuración del entorno para los comandos de terminal
\lstset{
  basicstyle=\ttfamily\footnotesize,    % Usa la tipografía de terminal
  breaklines=true,         % Permite saltos de línea
  frame=single,            % Añade un marco alrededor del código
  backgroundcolor=\color{gray!10}, % Fondo gris claro (opcional)
  literate={~}{{\textasciitilde}}1 % Corrige el carácter `~`
}


\author{natetas}
\title{Sobre Git y GitHub}
\date{}

\begin{document}


\maketitle

\tableofcontents
\newpage
\begin{center}
	\includegraphics[scale=0.3]{portada.jpg}  
\end{center}
\newpage
\chapter{Definiciones}
\newpage
\section{Git}
Es un sistema de control de versiones que te permite rastrear cambios en archivos a lo largo del tiempo. Te permite trabajar en diferentes versiones de un proyecto sin perder el historial de cambios.
\section{GitHub}
Es una plataforma en la nube que aloja \textbf{\textit{repositorios}} \textbf{\textit{Git}}. Permite almacenar proyectos en línea, colaborar con otros y acceder a los archivos desde cualquier computadora.
\section{Repositorio (repo)}
Es una carpeta que contiene todos los archivos del proyecto, junto con el historial de cambios. Puede estar de manera local o en \textbf{\textit{GitHub}} (remoto).
\section{Commit}
Es un \textit{"punto de guardado"} en el historial del proyecto. Cada vez que haces un \textbf{\textit{commit}}, guardas una versión de los archivos en ese momento.
\section{Ramas (branches)}
Es una línea de desarrollo independiente dentro de un \textbf{\textit{repositorio}}.

Cada \textbf{\textit{rama}} tiene su propio historial de \textbf{\textit{commits}}, lo que permite trabajar en nuevas características, experimentos o correcciones de errores sin afectar la versión principal del proyecto.

\subsection{Rama principal (main o master)}
Es la \textbf{\textit{rama}} predeterminada y suele contener la versión estable y funcional del proyecto.

Todos los cambios importantes (como nuevas versiones) se fusionan en esta \textit{\textbf{rama}}.

\subsection{Ramas secundarias}
Son \textbf{\textit{ramas}} que se crean a partir de la \textbf{\textit{rama principal}} (o de otra \textbf{\textit{rama}}) para trabajar en tareas específicas.

\newpage

\section{Clonar (clone)}
Descargar un repositorio remoto (de \textbf{\textit{GitHub}}) a una computadora local.

\section{Pull}
Traer cambios desde un \textbf{\textit{repositorio remoto}} a un repositorio local.

\section{Push}
Enviar cambios desde un \textbf{\textit{repositorio local}} a un repositorio remoto.

\section{Clave SSH}
Una \textbf{\textit{clave SSH}} es un método de autenticación seguro que permite a una computadora comunicarse con servicios remotos (como GitHub) sin necesidad de usar contraseñas. Consiste en un par de archivos:
\begin{itemize}
	\item \textbf{Clave privada:} Es un archivo que se guarda solo en tu computadora. Nunca debes compartirla, ya que es como una llave maestra que da acceso a tus repositorios.
	\item \textbf{Clave pública:} Es un archivo que puedes compartir con servicios como GitHub. GitHub usa esta clave para verificar que tienes la clave privada correspondiente.
\end{itemize}

\section{Archivo .md}
Un \textbf{\textit{archivo .md}} es un archivo de texto plano que usa la \textbf{\textit{sintaxis Marrkdown}}

\section{Sintaxis de Markdown}
\textbf{\textit{Markdown}} es un lenguaje de marcado ligero que permite formatear texto de manera sencilla, como añadir títulos, listas, enlaces, imágenes, código, etc., sin necesidad de usar HTML o herramientas complejas.

\newpage
\chapter{Comandos}
\newpage

\section{Navegación y manejo de archivos.}
\begin{tabularx}{\textwidth}{|l|X|} % Ajusta la tabla al ancho de la página
\hline
\textbf{Comando} & \textbf{Descripción} \\ \hline
\texttt{pwd} & Muestra la ruta del directorio actual. \\ \hline
\texttt{ls} & Lista los archivos y carpetas en el directorio actual. \\ \hline
\texttt{ls -l} & \makecell[l]{Lista los archivos y carpetas en formato largo \\ (con detalles como permisos y tamaño).} \\ \hline
\texttt{ls -a} & \makecell[l]{Lista todos los archivos, incluyendo los ocultos \\ (que empiezan con \texttt{.}).} \\ \hline
\texttt{cd <carpeta>} & Cambia al directorio especificado. \\ \hline
\texttt{cd ..} & Sube un nivel en la estructura de directorios. \\ \hline
\texttt{cd \textasciitilde} & Cambia al directorio home del usuario. \\ \hline
\texttt{mkdir <nombre>} & Crea una nueva carpeta con el nombre especificado. \\ \hline
\texttt{rmdir <nombre>} & Elimina una carpeta vacía. \\ \hline
\texttt{touch <archivo>} & \makecell[l]{Crea un archivo vacío o actualiza \\ su fecha de modificación.} \\ \hline
\texttt{cp <origen> <destino>} & Copia un archivo o carpeta. \\ \hline
\texttt{mv <origen> <destino>} & \makecell[l]{Mueve o renombra un archivo \\ o carpeta.} \\ \hline
\texttt{rm <archivo>} & Elimina un archivo. \\ \hline
\texttt{rm -r <carpeta>} & \makecell[l]{Elimina una carpeta y su contenido \\ de forma recursiva.} \\ \hline
\texttt{cat <archivo>} & Muestra el contenido de un archivo. \\ \hline
\texttt{less <archivo>} & \makecell[l]{Muestra el contenido de un archivo \\ página por página.} \\ \hline
\texttt{echo "texto"} & \makecell[l]{Muestra texto en la terminal \\ o lo redirige a un archivo.} \\ \hline
\texttt{echo "texto" > <archivo>} & \makecell[l]{Guarda el texto en un archivo \\ (sobrescribe).} \\ \hline
\texttt{echo "texto" >> <archivo>} & \makecell[l]{Añade el texto al final \\ de un archivo.} \\ \hline
\end{tabularx}
\newpage
\section{Manejo de repositorios de GitHub.}
\begin{table}[h!]
\centering
\begin{tabular}{|l|l|}
\hline
\textbf{Comando} & \textbf{Descripción} \\ \hline
\texttt{git init} & Inicializa un repositorio Git. \\ \hline
\texttt{git clone <URL>} & Clona un repositorio remoto. \\ \hline
\texttt{git status} & Muestra el estado del repositorio. \\ \hline
\texttt{git add <archivo>} & Añade archivos al área de preparación. \\ \hline
\texttt{git commit -m "Mensaje"} & Guarda los cambios en el historial. \\ \hline
\texttt{git push origin <rama>} & Sube cambios al repositorio remoto. \\ \hline
\texttt{git pull origin <rama>} & Trae cambios desde el repositorio remoto. \\ \hline
\texttt{git branch <nombre>} & Crea una nueva rama. \\ \hline
\texttt{git checkout <rama>} & Cambia a una rama específica. \\ \hline
\texttt{git merge <rama>} & Fusiona una rama con la rama actual. \\ \hline
\texttt{git log} & Muestra el historial de commits. \\ \hline
\texttt{git diff} & Muestra diferencias entre cambios. \\ \hline
\texttt{git remote add origin <URL>} & Añade un repositorio remoto. \\ \hline
\texttt{git tag <nombre>} & Crea una etiqueta (tag). \\ \hline
\texttt{git reset <archivo>} & Elimina un archivo del área de preparación. \\ \hline
\texttt{git checkout -- <archivo>} & Descarta cambios en un archivo. \\ \hline
\texttt{git remote -v} & Muestra los repositorios remotos configurados. \\ \hline
\texttt{git branch -d <rama>} & Elimina una rama local. \\ \hline
\end{tabular}
\caption{Comandos útiles de Git}
\label{tab:comandos-git}
\end{table}

\chapter{Para acceder a los repositorios.}
\newpage
\section{Antes de acceder.}
Sigue estos pasos para generar una \textbf{clave SSH} en tu computadora y añadirla a GitHub:
\begin{enumerate}
	\item \textbf{Genera la clave SSH}
	\begin{itemize}
		\item Abre la terminal en tu laptop.

		\item Ejecuta el siguiente comando para generar una \textbf{clave SSH}: 
		\begin{lstlisting}
ssh-keygen -t ed25519 -C "tu-email@ejemplo.com"
		\end{lstlisting}
		
		\item Presiona \textbf{Enter} para aceptar la ubicación predeterminada de la clave.
	\end{itemize}
	\item \textbf{Añade la clave SSH a tu agente SSH}
	\begin{itemize}
		\item Ejecuta los siguientes comandos para asegurarte de que el agente SSH esté activo y añadir tu clave:
		\begin{lstlisting}
eval "$(ssh-agent -s)"
ssh-add ~/.ssh/id_ed25519
		\end{lstlisting}
	\end{itemize}
	\item \textbf{Copia la clave pública}
	\begin{itemize}
		\item Usa el siguiente comando para mostrar tu clave pública en la terminal:
		\begin{lstlisting}
cat ~/.ssh/id_ed25519.pub
		\end{lstlisting}
		Verás algo como
		\begin{lstlisting}
ssh-ed25519 AAAAC3Nz...Z5Z5Z5Z5Z5  tu-email@ejemplo.com
		\end{lstlisting}
		Selecciona \textbf{el texto completo} (desde  \texttt{ssh-ed25519} hasta tu correo) y cópialo al portapapeles.
	\end{itemize}
	
	\item \textbf{Añade la clave pública a GitHub}
	\begin{itemize}
		\item Ve a la pestaña \textbf{SSH and GPG keys} en GitHub (dentro de Settings).

		\item Haz clic en \textbf{New SSH key.}

		\item Dale un título descriptivo (por ejemplo, "Mi Laptop").

		\item Pega la clave pública en el campo \textbf{Key.}

		\item Haz clic en \textbf{Add SSH key.}
	\end{itemize}
	\item \textbf{Verifica la conexión SSH}
	\begin{itemize}
		\item En la terminal, ejecuta:
		\begin{lstlisting}
ssh -T git@github.com
		\end{lstlisting}
		\item Si todo está configurado correctamente, verás un mensaje como:
		\begin{lstlisting}
Hi ElNatitis! You've successfully authenticated, but GitHub does not provide shell access.
		\end{lstlisting}
	\end{itemize}
\end{enumerate}
\newpage
\section{Para clonar el repositorio.}
Una vez que se configuró la \textbf{clave SSH} en tu computadora y esta se añadió a GitHub (Sección 3.1), puedes acceder a tus repositorios, usando SSH, siguiendo los siguientes pasos.


\begin{enumerate}
	\item \textbf{Obtén la URL SSH de tu repositorio}
	\begin{itemize}
		\item Ve a la página de tu repositorio en GitHub.

		\item Haz clic en el botón verde "Code".

		\item Selecciona la pestaña SSH y copia la URL. Se verá algo como esto:
		\begin{lstlisting}
git@github.com:ElNatitis/oach.git
		\end{lstlisting}
	\end{itemize}
	\item \textbf{Clona de repositorio}
	\begin{itemize}
		\item Abre la terminal y ejecuta el siguiente comando:
		\begin{lstlisting}
git clone git@github.com:ElNatitis/oach.git
		\end{lstlisting}
		\item Esto descargará el repositorio en una carpeta con el nombre del repositorio (oach en este caso).
	\end{itemize}
	\item \textbf{Navega a la carpeta del repositorio}
	\begin{itemize}
		\item Usa el siguiente comando para entrar a la carpeta del repositorio:
		\begin{lstlisting}
cd oach
		\end{lstlisting}
	\end{itemize}
\end{enumerate}



\end{document}
