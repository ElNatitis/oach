\documentclass[11pt]{beamer}
\usetheme{Warsaw}
\usepackage[utf8]{inputenc}
\usepackage[spanish]{babel}
\usepackage{amsmath}
\usepackage{amsfonts}
\usepackage{amssymb}
\usepackage{graphicx}
\author{N.Pérez-Medina}
\title{\textbf{Autocorrelación Parcial.}}
\setbeamercovered{transparent} 
\setbeamertemplate{navigation symbols}{} 
\logo{\includegraphics[scale=0.04]{logo_fing_uach_bn.png} } 
\institute{\textbf{FING - UACH}
	\\
	\begin{small}
	\textit{Series de tiempo}\\
	\textbf{Catedrático:} M. C. Erick N. Grijalva 
	\end{small}		
	} 
\date{\today} 
\subject{Series de tiempo} 


% Para tener un morado coqueto
\definecolor{lightPurple}{RGB}{186, 146, 209} % Morado claro
\definecolor{darkPurple}{RGB}{98, 54, 150}    % Morado oscuro
\setbeamercolor{structure}{fg=darkPurple} % Color principal para estructuras
\setbeamercolor{frametitle}{bg=darkPurple, fg=white} % Título del marco
\setbeamercolor{block title}{bg=darkPurple, fg=white} % Título de bloques
\setbeamercolor{block body}{bg=lightPurple!30, fg=black} % Cuerpo de bloques



\begin{document}

\begin{frame}
\titlepage
\end{frame}

\begin{frame}
\tableofcontents
\end{frame}

\section{Generalidades importantes.}
\begin{frame} 
\tableofcontents[currentsection] % Solo muestra la sección actual
\end{frame}

\begin{frame}{Definición}
\begin{block}{}
	Los \textbf{Modelos autorregresivos (AR)} son un tipo básico de \textit{modelos de procesos estacionarios.}
\end{block}
\end{frame}

\begin{frame}{Definición}
\begin{block}{}
	Los \textbf{Modelos autorregresivos (AR)} son un tipo básico de \textit{modelos de procesos estacionarios.}\footnote{Series de tiempo cuyas \textit{propiedades estadísticas} (media, varianza y autocorrelación) se mantienen \textbf{constantes} en el tiempo.}
\end{block}
\end{frame}

\begin{frame}{Premisa general.}
\begin{block}{}
	Los \textbf{valores actuales} dependen \textit{en cierta medida} de \textbf{valores previos.}
\end{block}

\end{frame}

\begin{frame}{Para cuantificar esa dependencia.}
\begin{block}{}
	\begin{itemize}
		\item \textbf{Función de autocovarianza}
		\item \textbf{Función de autocorrelación} 
	\end{itemize}
\end{block}
\end{frame}

\begin{frame}{Para cuantificar esa dependencia.}
\begin{block}{}
	\begin{itemize}
		\item \textbf{Función de autocovarianza:} mide la relación entre valores de la serie en distintos momentos del tiempo.
		\item \textbf{Función de autocorrelación} 
	\end{itemize}
\end{block}
\end{frame}

\begin{frame}{Para cuantificar esa dependencia.}
\begin{block}{}
	\begin{itemize}
		\item \textbf{Función de autocovarianza:} mide la relación entre valores de la serie en distintos momentos del tiempo.
		\item \textbf{Función de autocorrelación:} normaliza la autocovarianza para expresar la relación en términos de correlación, facilitando su interpretación.  
	\end{itemize}
\end{block}
\end{frame}

\begin{frame}{En resumen.}
\begin{block}{Premisa general.}
	Los \textbf{valores actuales} dependen \textit{en cierta medida} de \textbf{valores previos.}
\end{block}
\begin{block}{Para cuantificar esa dependencia.}
	\begin{itemize}
		\item \textbf{Función de autocovarianza:} mide la relación entre valores de la serie en distintos momentos del tiempo.
		\item \textbf{Función de autocorrelación:} normaliza la autocovarianza para expresar la relación en términos de correlación, facilitando su interpretación.  
	\end{itemize}
\end{block}
\end{frame}

\begin{frame}{\textit{Observación}}
\begin{block}{}
	Ambas funciones \textit{extienden} los conceptos clásicos de covarianza y correlación.
\end{block}

\begin{block}{Pero}
	En lugar de aplicarse entre dos variables diferentes, estamos modelando la relación lineal entre valores pasados y presentes.
\end{block}
\end{frame}

\section{Procesos autorregresivos}
\begin{frame} 
\tableofcontents[currentsection] % Solo muestra la sección actual
\end{frame}

\subsection{Proceso autorregresivo de primer orden AR(1)}
\begin{frame}{\textbf{AR(1)}}
\begin{block}{Definición}
Diremos que una serie sigue un \textit{Proceso autoregresivo de primer orden}, si su variabilidad puede describirse a través de: 
\begin{equation}
	z_t = c + \phi z_{t-1} + a_t
\end{equation}
con $-1 < \phi < 1$
\end{block}

\begin{block}{Donde:}
\begin{itemize}
	\item $c$: Captura el \textbf{efecto de la media en la serie.}
	\item $\phi$: Coeficiente que determina \textbf{la influencia de los valores pasados en el valor presente.}
	\item $a_t$: Término de \textbf{ruido blanco}
\end{itemize}
\end{block}
\end{frame}

\subsection{Proceso autorregresivo de segundo orden AR(2)}
\begin{frame}{\textbf{AR(2)}}
\begin{block}{Definición}
Diremos que una serie sigue un \textit{Proceso autoregresivo de segundo orden}, si su variabilidad puede describirse a través de: 
\begin{equation}
	z_t = c + \phi_1 z_{t-1} + \phi_2 z_{t-2} + a_t
\end{equation}
\end{block}

\begin{block}{Donde:}
\begin{itemize}
	\item $c$: Captura el \textbf{efecto de la media en la serie.}
	\item $\phi_1$ y $\phi_2$: Coeficientes que determinan \textbf{la influencia de los valores pasados en el valor presente.}
	\item $a_t$: Término de \textbf{ruido blanco}
\end{itemize}
\end{block}
\end{frame}
\subsection{Proceso autoregresivo general AR(p)}
\begin{frame}{\textbf{AR(p)}}
\begin{block}{Definición}
Diremos que una serie sigue un \textit{Proceso autoregresivo de orden} $p$, si su variabilidad puede describirse a través de: 
\begin{equation}
	z_t = c + \phi_1 z_{t-1} + \phi_2 z_{t-2} + \dots + \phi_p z_{t-p} + a_t
\end{equation}
\end{block}

\begin{block}{Donde:}
\begin{itemize}
	\item $c$: Captura el \textbf{efecto de la media en la serie.}
	\item $\phi_p$: Coeficientes que determinan \textbf{la influencia de los valores pasados en el valor presente.}
	\item $a_t$: Término de \textbf{ruido blanco}
\end{itemize}
\end{block}
\end{frame}

\begin{frame}{Para hacerlo más elegante.}
\begin{block}{Tomamos la esperanza de ambos lados.}
$$ E[z_t] = E[c + \phi_1 z_{t-1} + \phi_2 z_{t-2} + \dots + \phi_p z_{t-p} + a_t] $$
\end{block}
\begin{block}{Usando la propiedad lineal de la esperanza.}
$ E[z_t] = E[c] + \phi_1 E[z_{t-1}] + \phi_2 E[z_{t-2}] + \dots + \phi_p E[z_{t-p}] + E[a_t] $
\end{block}
\end{frame}

\begin{frame}{Hay que tomar en cuenta que}
\begin{block}{}
Como $a_t$ es \textbf{ruido blanco} tenemos que:
\begin{equation}
	E[a_t] = 0
\end{equation}
Dado que la serie es \textbf{estacionaria} tenemos que:
\begin{equation}
	E[z_t] = E[z_{t-1}] + E[z_{t-2}] + \dots + E[z_{t-p}] = \mu
\end{equation}
\end{block}
\end{frame}

\begin{frame}{Para hacerlo más elegante.}
\begin{block}{Tomamos la esperanza de ambos lados.}
$$ E[z_t] = E[c + \phi_1 z_{t-1} + \phi_2 z_{t-2} + \dots + \phi_p z_{t-p} + a_t] $$
\end{block}
\begin{block}{Usando la propiedad lineal de la esperanza.}
$ E[z_t] = E[c] + \phi_1 E[z_{t-1}] + \phi_2 E[z_{t-2}] + \dots + \phi_p E[z_{t-p}] + E[a_t] $
\end{block}
\begin{block}{Tomando en cuenta $(4)$ y $(5)$}
$$ \mu = c + \phi_1 \mu + \phi_2 \mu + \dots + \phi_p \mu $$
luego 
\begin{equation}
	c = \mu ( 1 - \phi_1 - \phi_2 - \dots - \phi_p )
\end{equation}

\end{block}
\end{frame}

\begin{frame}{Volviendo a nuestra definición}
\begin{block}{}
	$$ z_t = c + \phi_1 z_{t-1} + \phi_2 z_{t-2} + \dots + \phi_p z_{t-p} + a_t $$
\end{block}

\end{frame}

\begin{frame}{Volviendo a nuestra definición}
\begin{block}{}
	$$ z_t = c + \phi_1 z_{t-1} + \phi_2 z_{t-2} + \dots + \phi_p z_{t-p} + a_t $$
\end{block}
\begin{block}{Aplicando $\tilde{z_t}=z_t-\mu$}
	$ \tilde{z_t} + \mu = c + \mu + \phi_1 (\tilde{z}_{t-1} + \mu) + \phi_2 (\tilde{z}_{t-2} + \mu) + \dots + \phi_p (\tilde{z}_{t-p} + \mu) + a_t $
\end{block}

\end{frame}

\begin{frame}{Volviendo a nuestra definición}
\begin{block}{}
	$$ z_t = c + \phi_1 z_{t-1} + \phi_2 z_{t-2} + \dots + \phi_p z_{t-p} + a_t $$
\end{block}
\begin{block}{Aplicando $\tilde{z_t}=z_t-\mu$}
	$ \tilde{z_t} + \mu = c + \mu + \phi_1 (\tilde{z}_{t-1} + \mu) + \phi_2 (\tilde{z}_{t-2} + \mu) + \dots + \phi_p (\tilde{z}_{t-p} + \mu) + a_t $
\end{block}
\begin{block}{Reagrupando términos}
	$ \tilde{z_t}  = \phi_1 \tilde{z}_{t-1} + \phi_2 \tilde{z}_{t-2} + \dots + \phi_p \tilde{z}_{t-p} + a_t + c - \mu ( 1 - \phi_1 - \phi_2 - \dots - \phi_p )$
\end{block}
\begin{block}{Recordando $(6)$}
\begin{equation}
\tilde{z_t}  = \phi_1 \tilde{z}_{t-1} + \phi_2 \tilde{z}_{t-2} + \dots + \phi_p \tilde{z}_{t-p} + a_t
\end{equation}
\end{block}
\end{frame}

\begin{frame}{\textbf{AR(p)}}
\begin{block}{Definición}
Diremos que una serie sigue un \textit{Proceso autoregresivo de orden} $p$, si su variabilidad puede describirse a través de: 
$$ \tilde{z_t}  = \phi_1 \tilde{z}_{t-1} + \phi_2 \tilde{z}_{t-2} + \dots + \phi_p \tilde{z}_{t-p} + a_t $$
\end{block}

\begin{block}{Donde:}
\begin{itemize}
	\item $\tilde{z_t}=z_t-\mu$
	\item $\phi_p$: Coeficientes que determinan \textbf{la influencia de los valores pasados en el valor presente.}
	\item $a_t$: Término de \textbf{ruido blanco}
\end{itemize}
\end{block}
\end{frame}

\section{Función de autoccorelación simple}
\begin{frame} 
\tableofcontents[currentsection] % Solo muestra la sección actual
\end{frame}

\begin{frame}{¿Qué hace?}
\begin{block}{}
	Mide qué tan parecido es $z_t$​ con $z_{t - k}$ para distintos rezagos $k$.
\end{block}
\end{frame}

\begin{frame}{¿Qué desventajas presenta?}
\begin{block}{}
	Esas medidas consideran tanto relaciones \textbf{directas} como \textbf{indirectas}.
\end{block}
\end{frame}

\begin{frame}{Eso quiere decir...}
\begin{block}{}
Si $z_t$ está fuertemente correlacionado con $z_{t-1}$ y​ $z_{t-1}$ también está correlacionado con $z_{t-2}$, entonces $z_t$ y $z_{t-2}$ \textbf{pueden aparecer como correlacionados, aunque no haya una relación directa entre ellos.}
\end{block}
\end{frame}


\section{Funcion de autocorrelación parcial}
\begin{frame} 
\tableofcontents[currentsection] % Solo muestra la sección actual
\end{frame}
\subsection{¿Qué hace?}
\begin{frame}{¿Qué hace?}
\begin{block}{}
	Mide qué tan parecido es $z_t$​ con $z_{t - k}$ \textbf{eliminando la influencia de los rezagos intermedios.}
\end{block}
\end{frame}

\begin{frame}{¿Qué ventajas presenta?}
\begin{block}{}
\begin{itemize}
	\item Si el modelo real es \textbf{AR(2)}, la \textbf{\textit{PACF}} será \textbf{significativa solo en los rezagos 1 y 2}, y cercana a 0 para $k>2$.
	\item En cambio, la \textbf{ACF} podría mostrar una correlación decreciente incluso para rezagos más grandes, porque incluye influencias indirectas.
\end{itemize}
\end{block}
\end{frame}

\subsection{¿Cómo lo hace?}
\begin{frame}{Con 3 sencillos pasos}
\begin{block}{Para calcular la \textbf{ACP} entre un instante $z_t$ y uno $z_{t-k}$ }
\begin{enumerate}
	\item Regresión de $\tilde{z}_t$ contra sus valores intermedios
	\item Regresión de $\tilde{z}_{t-k}$ contra sus valores intermedios
	\item Correlación entre $u$ y $v$
\end{enumerate}
\end{block}
\end{frame}

\begin{frame}{1. Regresión de $\tilde{z}_t$ contra sus valores intermedios}
\begin{block}{Aplicando regresión encontraremos}
$$ \tilde{z_t}  = \beta_1 \tilde{z}_{t-1} + \beta_2 \tilde{z}_{t-2} + \dots + \beta_{k-1} \tilde{z}_{t-k} + u_t $$
\end{block}
\begin{block}{¿Qué representa $u_t$?}
	La parte de $\tilde{z_t}$ que no está explicada por los valores intermedios.
\end{block}
\end{frame}

\begin{frame}{2. Regresión de $\tilde{z}_{t-k}$ contra sus valores intermedios}
\begin{block}{Aplicando regresión encontraremos}
$$ \tilde{z}_{t-k}  = \gamma_1 \tilde{z}_{t-1} + \gamma_2 \tilde{z}_{t-2} + \dots + \gamma_{k-1} \tilde{z}_{t-k+1} + v_t $$
\end{block}
\begin{block}{¿Qué representa $v_t$?}
	La parte de $\tilde{z}_{t-k}$ que no está explicada por los valores intermedios.
\end{block}
\end{frame}

\begin{frame}{Correlación entre $u$ y $v$}
\begin{block}{Tomando en cuenta lo anterior}
La correlación entre $u$ y $v$ mide \textbf{\textit{únicamente}} la relación directa entre $\tilde{z}_t$ y $\tilde{z}_{t-k}$, \textbf{eliminando la influencia de los valores intermedios.}
\begin{equation}
	\rho = Corr(u_t,v_t)
\end{equation}
$\rho$ es precisamente la autocorrelación parcial en el rezago $k$.
\end{block}
\end{frame}

\begin{frame}{Idea gráfica (1/2)}
\centering
\includegraphics[scale=0.5]{idea-grafica-1.png} 
\end{frame}

\begin{frame}{Idea gráfica (2/2)}
\centering
\includegraphics[scale=0.35]{idea-grafica-2.png} 
\end{frame}

\section*{Gracias.}
\begin{frame} 
\tableofcontents[currentsection] % Solo muestra la sección actual
\end{frame}

\begin{frame}{Por su atención:}
\centering
\Huge
\textbf{Gracias y bonito vierneeees}
\end{frame}


\end{document}
