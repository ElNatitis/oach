\documentclass[11pt]{beamer}
\usetheme{Warsaw}
\usepackage[utf8]{inputenc}
\usepackage[spanish]{babel}
\usepackage{amsmath}
\usepackage{amsfonts}
\usepackage{amssymb}
\usepackage{graphicx}
\author{N.Perez-Medina}
\title{Autocorrelación Parcial.}
\setbeamercovered{transparent} 
\setbeamertemplate{navigation symbols}{} 
\logo{} 
\institute{FING - UACH} 
\date{\today} 
\subject{Series de tiempo} 


% Para tener un morado coqueto
\definecolor{lightPurple}{RGB}{186, 146, 209} % Morado claro
\definecolor{darkPurple}{RGB}{98, 54, 150}    % Morado oscuro
\setbeamercolor{structure}{fg=darkPurple} % Color principal para estructuras
\setbeamercolor{frametitle}{bg=darkPurple, fg=white} % Título del marco
\setbeamercolor{block title}{bg=darkPurple, fg=white} % Título de bloques
\setbeamercolor{block body}{bg=lightPurple!30, fg=black} % Cuerpo de bloques



\begin{document}

\begin{frame}
\titlepage
\end{frame}

\begin{frame}{Para tener en cuenta.}
\centeting
\textit{Los procesos autoregresivos \textbf{extienden} el concepto de regresión.}
\end{frame}

\begin{frame}
\tableofcontents
\end{frame}

\section{Procesos autorregresivos}
\begin{frame}{Generalidades importantes.}
\begin{block}{Premisa general.}
	Los \textbf{valores actuales} dependen \textit{en cierta medida} de \textbf{valores previos.}
\end{block}

\end{frame}

\begin{frame}{Generalidades importantes.}
\begin{block}{Premisa general.}
	Los \textbf{valores actuales} dependen \textit{en cierta medida} de \textbf{valores previos.}
\end{block}

\begin{block}{Para cuantificar esa dependencia.}
	\begin{itemize}
		\item \textbf{Función de autocovarianza}
		\item \textbf{Función de autocorrelación} 
	\end{itemize}
\end{block}
\end{frame}

\begin{frame}{Generalidades importantes.}
\begin{block}{Premisa general.}
	Los \textbf{valores actuales} dependen \textit{en cierta medida} de \textbf{valores previos.}
\end{block}

\begin{block}{Para cuantificar esa dependencia.}
	\begin{itemize}
		\item \textbf{Función de autocovarianza:} mide la relación entre valores de la serie en distintos momentos del tiempo.
		\item \textbf{Función de autocorrelación} 
	\end{itemize}
\end{block}
\end{frame}

\begin{frame}{Generalidades importantes.}
\begin{block}{Premisa general.}
	Los \textbf{valores actuales} dependen \textit{en cierta medida} de \textbf{valores previos.}
\end{block}

\begin{block}{Para cuantificar esa dependencia.}
	\begin{itemize}
		\item \textbf{Función de autocovarianza:} mide la relación entre valores de la serie en distintos momentos del tiempo.
		\item \textbf{Función de autocorrelación:} normaliza la autocovarianza para expresar la relación en términos de correlación, facilitando su interpretación.  
	\end{itemize}
\end{block}
\end{frame}

\begin{frame}{\textit{Extender}}
\begin{block}{}
	Ambas funciones \textit{extienden} los conceptos clásicos de covarianza y correlación.
\end{block}

\begin{block}{Pero}
	En lugar de aplicarse entre dos variables diferentes, las utilizan para analizar la relación de una misma serie a lo largo del tiempo.
\end{block}

\end{frame}

\subsection{Proceso autorregresivo de primer orden AR(1)}
\begin{frame}{\textbf{AR(1)}}
\begin{block}{Definición}
Diremos que una serie sigue un \textit{Proceso autoregresivo de primer orden}, si su variabilidad puede describirse a través de: 
\begin{equation}
	z_t = c + \phi z_{t-1} + a_t
\end{equation}
con $-1 < \phi < 1$
\end{block}

\begin{block}{Donde:}
\begin{itemize}
	\item $c$: Captura el \textbf{efecto de la media en la serie.}
	\item $\phi$: Coeficiente que determina \textbf{la influencia de los valores pasados en el valor presente.}
	\item $a_t$: Término de \textbf{ruido blanco}
\end{itemize}
\end{block}


\end{frame}

\subsection{Proceso autorregresivo de primer orden AR(2)}
\subsection{Proceso autoregresivo general AR(p)}
\section{Función de autoccorelación simple}
\section{Funcion de autocorrelación parcial}
\begin{frame}{•}

\end{frame}

\end{document}
