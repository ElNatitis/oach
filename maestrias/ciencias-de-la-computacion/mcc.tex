\documentclass[10pt,a4paper]{book}
\usepackage[utf8]{inputenc}
\usepackage{amsmath}
\usepackage{amsfonts}
\usepackage{amssymb}
\usepackage{graphicx}
\usepackage{hyperref}

\author{nati}
\title{\textbf{Maestría en Ciencias con Especialidad en Computación y Matemáticas Industriales}}
\date{\today}


\begin{document}

\maketitle
\newpage
\tableofcontents
\chapter{Admisión al Programa.}
\section{Estar titulados a más tardar en julio del año en que desean ingresar.}

El proceso para que seas admitido requiere que estés titulada o titulado a más tardar en julio del año en que deseas ingresar, o bien presentar cartas de tu asesor y autoridades académicas que hagan constar que la tesis se encuentra en etapa de revisión y que el examen de grado se presentará dentro de los primeros tres meses del semestre escolar de ingreso.
En caso de obtener la titulación por promedio o créditos de maestría, debes presentar constancia. El promedio mínimo de licenciatura para ingresar a un posgrado del CIMAT es de 8 (ocho).

\section{Elaborar la solicitud según la guía de la convocatoria.}

La solicitud se elabora digitalmente en el siguiente link.

\href{https://posgrados.cimat.mx/ControlEscolar/web/Admisiones/SolicitudAdmision}{\textbf{Solicitud de admisión}}

\section{Aprobar el examen escrito de admisión.}

El examen escrito para la admisión a la \textbf{Maestría en Ciencias de la Computación} consiste en dos partes:
\begin{enumerate}
	\item \textbf{Un examen de matemáticas}, cuyo fin es:
	\begin{enumerate}
		\item Evaluar los conocimientos del candidato en álgebra lineal, cálculo, geometría analítica, y lógica
		\item Comprobar su capacidad de análisis de problemas simples y de formulación en un lenguaje matemático para su resolución. 
	\end{enumerate}
	\item \textbf{Un examen de programación}, en el cual buscamos evaluar los conocimientos sobre los fundamentos de la programación, los elementos más básicos de los lenguajes de programación y las nociones más elementales de estructuras de datos y algoritmos.
\end{enumerate}

\section{Aprobar el mini-curso propedéutico.}

Los candidatos deben cursar un mini-curso de aproximadamente una semana, durante el cual se les continúa evaluando en programación y matemáticas.

\section{Atender una entrevista.}

Los candidatos se entrevistan con un grupo de investigadores del área de Ciencias de la Computación. Finalmente, con base en la decisión del Comité de Admisiones, la Coordinación de Formación Académica comunica los resultados del proceso de admisión a cada uno de los solicitantes. 

Los solicitantes aceptados deberán entregar todos los documentos requeridos para completar su registro.

\section{Firmar una carta-compromiso.}
La carta-compromiso es de dedicación exclusiva y de tiempo completo a las actividades de la Maestría en Ciencias de la Computación.

\newpage


\chapter{Examen escrito de admisión.}

\section{Temario examen de matemáticas.}
\subsection{Álgebra lineal.}
\begin{itemize}
	\item \textbf{Vectores y matrices}
	\begin{itemize}
		\item Adición de matrices 
		\item Multiplicación escalar
		\item Multiplicación de matrices
		\item Transpuesta de una matriz
	\end{itemize}		
	
	\item Ecuaciones lineales
	\begin{itemize}
		\item Sistemas de ecuaciones lineales
		\item Sistemas en forma triangular
		\item Eliminación Gaussiana
		\item Determinantes
		\item Sistemas homogéneos de ecuaciones lineales
	\end{itemize}
	\item Espacios vectoriales
	\begin{itemize}
		\item Combinaciones lineales
		\item Espacio generado
		\item Subespacios
	\end{itemize}		
	\item Independencia lineal
	\begin{itemize}
		\item Base y dimensión; 
		\item Rango de una matriz.
	\end{itemize}		
	\item Coordenadas en espacios vectoriales
	\begin{itemize}
		\item Cambio de base
	\end{itemize} 
	\item Producto interior
	\begin{itemize}
		\item Cauchy-Schwarz 
		\item Ortogonalidad
	\end{itemize}	   
\end{itemize}

\subsubsection{Referencias sugeridas}
\begin{enumerate}
	\item Stanley Grossman. \textit{Algebra lineal.} Mc Graw-Hill, 7th edition, 2012.
	\item Seymour Lipschutz and Marc Lipson. \textit{Beginning linear algebra.} Schaum’s Outline Series. Mc Graw-Hill, 5th edition, 2012.
\end{enumerate}

\newpage

\subsection{Cálculo}

\begin{itemize}
	\item Plano numérico; 
	\begin{itemize}
		\item Coordenadas y gráficas de ecuaciones
		\item Funciones algebraicas logarítmicas, exponenciales y trigonométricas.
	\end{itemize}		
	\item Límites y continuidad
	\begin{itemize}
		\item Interpretación geométrica.
	\end{itemize}
	\item Derivadas
	\begin{itemize}
		\item Regla de la cadena
		\item Derivadas de funciones trigonométricas
		\item Derivadas de funciones compuestas
		\item Derivadas de orden superior
		\item Máximos y mínimos
		\item Concavidad y punto de inflexión
		\item Funciones crecientes y decrecientes
		\item Gráficas de funciones
	\end{itemize}		
	\item Inversa de una función y su derivada
	\item Integral definida 
	\begin{itemize}
		\item Interpretación geométrica
		\item Teorema fundamental de cálculo
		\item Integración por partes.
	\end{itemize}
	\item Bases de integrales múltiples
	\begin{itemize}
		\item Derivadas parciales.
	\end{itemize}
	\item Bases de Ecuaciones Diferenciales Ordinarias.
\end{itemize}

\subsubsection{Referencias sugeridas.}

\begin{itemize}
	\item Jerome Keisler. \textit{Elementary calculus, an infinitesimal approach}. Dover Publications, 2nd edition, 2000. https://www.math.wisc.edu/~keisler/calc.html.
	\item Morris Kline. \textit{Calculus: An Intuitive and Physical Approach.} Dover Books on Mathematics. Dover Publications, 2nd edition, 1998.
	\item Jerrold Marsden and Alan Weinstein. \textit{Calculus I}. Undergraduate Texts in Mathematics. Springer, 2nd edition, 1985. http://authors.library.caltech.edu/25030/1/Calc1w.pdf.
	 \item Elliott Mendelson. \textit{Beginning Calculus}. Schaum’s Outline Series. Mc Graw-Hill, 6th edition, 2012.
	 \item Gilbert Strang.\textit{ Calculus.} Wellesley-Cambridge Press, 1991. http://cw.mit.edu/resources/res-18-001-calculus-online-textbook-spring-2005/textbook/.
	 \item George B. Thomas. \textit{Cálculo de un variable}. Pearson, 12th edition, 2010.
\end{itemize}
\newpage
\subsection{Geometría analítica.}
\begin{itemize}
	\item Puntos en el plano 
	\begin{itemize}
		\item Distancia
		\item Coordenadas rectangulares.
	\end{itemize}		
	\item Ecuación de una recta
	\begin{itemize}
		\item Intersecciones de rectas
		\item Angulos
		\item Recta tangente.
	\end{itemize}	
	\item Producto escalar
	\begin{itemize}
		\item Ortogonalidad
	\end{itemize}
	\item Ecuación de un círculo
	\begin{itemize}
		\item Ecuación de cónicas.
	\end{itemize}		
\end{itemize}

\subsubsection{Referencias sugeridas.}

\begin{itemize}
	\item Jim Hefferon. \textit{Linear Algebra.} -, 1996. http://joshua.smcvt.edu/linearalgebra/.
	\item Joseph Kindle. \textit{Teoria y problemas de Geometria Analitica Plana y del Espacio}. Serie de compendios Schaum. Mc Graw-Hill, 1970. http://adria.inaoep.mx/\%7Ediplomados/biblio/analitica/GAKindle.pdf.
	\item Charles Lehman. \textit{Geometróa Analítica.} Limusa, 1989. https://archive.org/details/GeometriaAnalitica.
\end{itemize}
\newpage
\subsection{Otros temas}

\begin{itemize}
	\item Combinatoria.	
	\item Lógica.
	\item Teoría de conjuntos.
	\item Demostración matemática: construcción, inducción, ...
\end{itemize}

\subsubsection{Referencias sugeridas.}
\begin{itemize}
	\item Seymour Lipschutz and Marc Lipson. \textit{Matemáticas Discretas.} Serie de compendios Schaum. Mc Graw - Hill, 3rd edition, 2009.
\end{itemize}
\newpage
\section{Temario examen de programación.}
Queremos resaltar que en el examen de programación, el énfasis no es tanto en la sintaxis del lenguaje utilizado, sino más bien en la estructura y la lógica interna del algoritmo empleado para resolver el problema. Los candidatos podrán usar el lenguaje de su elección para contestar a las preguntas; podrán también hacer uso de pseudo-código.

Los temas que podr´ıan aparecer en los problemas propuestos son los siguientes:
\begin{itemize}
	\item Tipos de datos. 
	\item Tipos enteros/tipos flotantes.
	\item Variables.
	\item Operadores básicos de asignación, de comparación; operadores lógicos; operadores aritméticos.
	\item Funciones.
	\item Estructuras de control: for, while, if.
	\item Estructuras de datos elementales: arrays (unidimensionales y multidimensionales), listas ligadas, pilas, colas.
	\item Recursividad. Búsqueda binaria.
	\item Algoritmos de ordenamiento básicos.
	\item Análisis de código básico.
\end{itemize}

\subsection{Referencias sugeridas.}
\begin{itemize}
\item [1] Thomas H. Cormen, Clifford Stein, Ronald L. Rivest, and Charles E. Leiserson. Introduction to Algorithms. McGraw-Hill Higher Education, 2nd edition, 2001.
\item [2] Bruce Eckel. Thinking in C++. Prentice Hall, 2nd edition, 2000. http://mindview.net/Books/TICPP/ThinkingInCPP2e.html.
\item [3] Stephen Kochan. Programming in C. Developer’s Library. Addison-Wesley, 4th edition, 2014.
\item [4] Robert Sedgewick and Kevin Wayne. Algorithms. Addison-Wesley, 4th edition, 2011.
\end{itemize}




\end{document}
