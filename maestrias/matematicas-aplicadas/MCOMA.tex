\documentclass[10pt,a4paper]{article}
\usepackage[utf8]{inputenc}
\usepackage[spanish]{babel}
\usepackage{amsmath}
\usepackage{amsfonts}
\usepackage{amssymb}
\usepackage{graphicx}
\usepackage{hyperref}
\usepackage[left=2cm,right=2cm,top=2cm,bottom=2cm]{geometry}

% Datos
\title{Maestría en Ciencias con Orientación en Matemáticas Aplicadas}
\author{natetas}
\date{}

\begin{document}
\maketitle
\newpage
\tableofcontents
\newpage
\section{Sobre el instituto.}
Los programas de Maestrías en Ciencias orientados a la investigación ofrecen varias áreas de especialidad, todas registradas en el Sistema Nacional de Posgrados del CONACHYT.

En el CIMAT los investigadores son quienes imparten clases como profesores de los programas de posgrado. La gran mayoría de ellos pertenecen al \textbf{Sistema Nacional de Investigadores (SNI).} Junto con los investigadores ordinarios, investigadores visitantes y posdoctorales que recibe el Centro regularmente, aportan a la formación de los estudiantes y extienden su perspectiva científica.

\section{Admisión al Programa.}
La admisión al Programa de Maestría en Ciencias con Orientación en Matemáticas Aplicadas se lleva a cabo anualmente. \footnote{Bajo circunstancias excepcionales, a juicio de los coordinadores respectivos, se considerarán admisiones en fechas distintas a las usuales. }Para ingresar el aspirante deberá cumplir con lo siguiente: 

\begin{enumerate}
	\item Cumplir los requisitos de admisión que indique el Reglamento General de Estudios de Posgrados (RGEP) de CIMAT.
	
	\item Presentar y aprobar el examen de admisión ante un comité de admisión designado por el \textit{Comité Académico del Posgrado (CAP).} En el examen para ingreso a la maestría, se valorará el \textbf{manejo eficiente de nociones básicas de Ecuaciones Diferenciales, Cálculo, Álgebra lineal, Estadística y Programación}, problemas de habilidad matemática, así como la motivación del aspirante hacia los estudios de posgrado en el área elegida.
	
	\item Presentarse a una entrevista de preselección ante un comité de admisión designado por el CAP.
	
	\item En base al desempeño académico destacado del solicitante, y bajo recomendación del comité de admisión, el CAP podrá convalidar la presentación del examen de admisión.  
\end{enumerate}
\newpage
\subsection{Temario del examen.}
\begin{enumerate}
	\item \textbf{Cálculo}
	\begin{enumerate}
		\item \textit{Geometría del espacio euclidiano.}
		\begin{enumerate}
			\item Producto interno.
			\item Vectores en el espacio tridimensional y producto cruz.
			\item Coordenadas esféricas y cilíndricas.
		\end{enumerate}
		\item \textit{Diferenciación.}
		\begin{enumerate}
			\item Límites, continuidad.
			\item Derivadas, derivadas parciales, regla del producto y regla de la cadena.
			\item Aproximación y polinomio de Taylor. Método de Newton.
			\item Problemas de optimización, ecuaciones de punto crítico, multiplicadores de Lagrange, criterio de la segunda derivada.
		\end{enumerate}
		\item \textit{Integración.}
		\begin{enumerate}
			\item Integral, interpretación geométrica y métodos de integración
			\begin{enumerate}
				\item Fracciones parciales
				\item Sustituciones trigonométricas
			\end{enumerate}
			\item Integral de línea, superficie, flujo y volumen. Fórmulas de cambios de variables.	 
			\item Teoremas fundamentales del cálculo
			\begin{enumerate}
				\item Divergencia
				\item Green
				\item Stokes
			\end{enumerate}
		\end{enumerate}
	\end{enumerate}
	\item \textbf{Álgebra lineal}
	\begin{enumerate}
		\item \textit{Ecuaciones Lineales}
		\begin{enumerate}
			\item Matrices y operaciones elementales
			\item Matrices escalón y solución de ecuaciones
			\item Producto de matrices, matrices invertibles
			\item Determinantes, interpretación geométrica y regla de Cramer
		\end{enumerate}
		\item \textit{Espacios vectoriales}
		\begin{enumerate}
			\item Independencia lineal
			\item Bases y dimensión
			\item Subespacio vectorial
		\end{enumerate}
		\item \textit{Transformaciones Lineales}
		\begin{enumerate}
			\item Núcleo y Rango
			\item Subespacios de matrices
			\item Cambio de bases
			\item Valores y vectores propios
		\end{enumerate}
		\item \textit{Ortogonalidad}
		\begin{enumerate}
			\item Proyecciones
			\item Gram-Schmidt
		\end{enumerate}
	\end{enumerate}
	\item \textbf{Ecuaciones diferenciales}
	\begin{enumerate}
		\item \textit{Ecuasiones de priemr orden}
		\begin{enumerate}
			\item Ecuaciones lineales
			\item Separación de variables
			\item Ecuaciones exactas
		\end{enumerate}
		\item \textit{Ecuaciones lineales de segundo orden}
		\begin{enumerate}
			\item Wronskiano e independencia lineal
			\item Reducción de orden
			\item Variación de parámetros
		\end{enumerate}
		\item \textit{Aplicaciones}
		\begin{enumerate}
			\item Problemas de mezclas
			\item Circuitos eléctricos
			\item Vibraciones mecánicas. Oscilador armónico
		\end{enumerate}
	\end{enumerate}
\end{enumerate}
\newpage
\section{Plan de estudios.}
\subsection{Semestre 1}
\subsubsection{Modelación Dinámica}

El curso introduce la teoría lineal de Ecuaciones Diferenciales Ordinarias. Después
de una introducción a las ecuaciones de primer orden, se presenta el Algebra Lineal Básica como en [1]. El Capítulo 3 sigue la exposición rigurosa de [2]. Con base en [1] la teoría cualitativa en el plano es el contenido del Capítulo 4. Los capítulos 5 y 6 siguen [2]. El 5 es una presentación auto contenida.


\textbf{Temas y subtemas.}


\begin{itemize}
    \item \textbf{1. Ecuaciones de primer orden}
    \begin{itemize}
        \item 1.1 Modelos en ecuaciones diferenciales
        \item 1.2 Ecuaciones autónomas
        \item 1.3 Análisis Cualitativo
    \end{itemize}

    \item \textbf{2. Ángulos Iniciales}
    \begin{itemize}
        \item 2.1 Conceptos básicos
        \item 2.2 Valores y vectores propios
        \item 2.3 Valores propios complejos
        \item 2.4 Bases y subespacios
        \item 2.5 Valores propios repetidos
        \item 2.6 Propiedades genéricas
    \end{itemize}

    \item \textbf{3. Ecuaciones Lineales}
    \begin{itemize}
        \item 3.1 La técnica matricial
        \item 3.2 Sistema lineal autónomo de primer orden
        \item 3.3 Ecuación lineal autónoma de orden n
        \item 3.4 Sistema general de primer orden
        \item 3.5 Ecuación lineal de orden n
        \item 3.6 Sistemas periódicos
        \item 3.7 Perturbación de sistemas lineales
        \item 3.8 Forma Canónica de Jordan
    \end{itemize}

    \item \textbf{4. Sistemas en el Plano}
    \begin{itemize}
        \item 4.1 Ecuaciones de segunda orden
        \item 4.2 Rectas fijas
        \item 4.3 El plano fase-determinante
        \item 4.4 Clasificación dinámica
    \end{itemize}

    \item \textbf{TEMAS COMPLEMENTARIOS}

    \item \textbf{5. Problemas con Valores en la Frontera}
    \begin{itemize}
        \item 5.1 Separación de variables
        \item 5.2 Operadores simétricos compactos
        \item 5.3 Ecuaciones de Sturm-Liouville
        \item 5.4 El problema regular de Sturm-Liouville
        \item 5.5 Teoría de oscilación
        \item 5.6 Ecuaciones periódicas de Sturm-Liouville
    \end{itemize}

    \item \textbf{6. El problema con Valores iniciales}
    \begin{itemize}
        \item 6.1 Teoría de Punto Fijo
        \item 6.2 Existencia y Unicidad
        \item 6.3 Dependencia de las condiciones iniciales
        \item 6.4 Perturbación Regular
    \end{itemize}
\end{itemize}

\newpage

\subsubsection{Métodos Numéricos}

Este es un curso clásico de métodos numéricos. Se cubren los temas básicos de álgebra lineal numérica y cálculo diferencial e integral numérico. Para completar la introducción al análisis numérico se presentan también algunos temas de aproximación y ecuaciones diferenciales. En los diferentes temas se buscará un balance entre la teoría detrás del método, su aplicación a problemas prácticos y su implementación computacional. Se presentarán soluciones numéricas utilizando cómputo en paralelo.

\textbf{Temas y subtemas}

\begin{itemize}
    \item \textbf{Introducción:}
    \begin{itemize}
        \item[a.] Preliminares
        \item[b.] Problemas no lineales en una variable.
        \begin{itemize}
            \item 1. Solución de ecuaciones: Bisección, método de Newton.
            \item 2. Minimización de funciones.
        \end{itemize}
    \end{itemize}

    \item \textbf{Álgebra lineal numérica:}
    \begin{itemize}
        \item[a.] Solución de sistemas lineales.
        \begin{itemize}
            \item 1. Eliminación Gaussiana, Sustitución hacia atrás.
            \item 2. Descomposición LU, QR, inversa y determinante de una matriz.
            \item 3. Métodos iterativos: Jacobi, Gauss-Seidel, gradiente conjugado.
            \item 4. Precondicionamiento de sistemas iterativos.
            \item 5. El problema de valores propios.
            \item 6. Método de Jacobi.
            \item 7. Método de la potencia.
            \item 8. El problema generalizado de valores propios.
            \item 9. Mínimos cuadrados.
        \end{itemize}
    \end{itemize}

    \item \textbf{Métodos numéricos en cálculo:}
    \begin{itemize}
        \item[a.] Interpolación.
        \begin{itemize}
            \item 1. Polinomios.
            \item 2. Splines cúbicos.
            \item 3. Elementos finitos.
            \item 4. Integración y diferenciación.
            \item 5. Métodos clásicos.
            \item 6. Método de Romberg.
            \item 7. Cuadratura de Gauss y polinomios ortogonales.
            \item 8. Diferencias finitas.
        \end{itemize}
        \item[b.] Problemas no lineales multivariados.
        \begin{itemize}
            \item 1. Sistemas no lineales. Métodos de Newton.
            \item 2. Minimización de funciones.
        \end{itemize}
    \end{itemize}

    \item \textbf{Ecuaciones Diferenciales:}
    \begin{itemize}
        \item[a.] Problemas con valores iniciales.
        \begin{itemize}
            \item 1. Método de Euler.
            \item 2. Métodos Runge-Kutta.
            \item 3. Otros métodos.
        \end{itemize}
        \item[b.] Problemas con valores a la frontera.
        \begin{itemize}
            \item 1. Diferencias finitas.
            \item 2. Elemento finito.
            \item 3. Problemas de advección-difusión.
            \item 4. Problemas de valores propios.
        \end{itemize}
    \end{itemize}
\end{itemize}

\newpage

\subsubsection{Modelos Estocásticos}

Al finalizar el curso los alumnos:
\begin{itemize}
	\item Habrán adquirido intuición sobre razonamiento probabilístico aplicado a fenómenos aleatorios.
	\item Conocerán conceptos generales de teoría de probabilidad, su concepción y su manejo matemático. Se promoverá el recurso de simulación en computadora de procesos estocásticos para fines didácticos y análisis de datos.
	\item Estarán familiarizados algunos con modelos probabilísticos de frecuente aplicación, así como para fenómenos aleatorios en espacio-tiempo enfatizando procesos de Poisson y Cadenas de Markov.
	\item Conocerán elementos de estimación paramétrica con base en muestras aleatorias y algunas herramientas de exploración de datos. Manejarán conceptos de probabilidad para espacios discretos a un nivel que presupone manejo de series pero no de Teoría de la Medida.
\end{itemize}


\textbf{Temas y subtemas}

\begin{itemize}
    \item \textbf{1. Elementos de probabilidad y estadística}
    \begin{itemize}
        \item 1.1 Experimentos aleatorios y espacios de probabilidad.
        \item 1.2 Leyes de probabilidad, independencia y probabilidad condicional.
        \item 1.3 Variables aleatorias, distribuciones discretas y continuas. Funciones de distribución, valor esperado. Momentos: Función generadora de momentos y función generadora de probabilidades.
        \item 1.4 Familias notables de distribuciones discretas y continuas: Bernoulli, binomial, Poisson, geométrica, binomial negativa, normal, exponencial.
        \item 1.5 Muestras aleatorias, independientes e idénticamente distribuidas. Momentos empíricos. La función de distribución empírica.
        \item 1.6 La función de verosimilitud y elementos de estimación paramétrica por el método de máxima verosimilitud.
        \item 1.7 La desigualdad de Chebyshev, convergencia en probabilidad y la Ley de los Grandes Números.
        \item 1.8 Métodos de estadística exploratoria. Conocimientos de momentos empíricos: Estimación paramétrica por el método de momentos. Estimación de densidades: Kernel y histogramas.
        \item 1.9 Vectores aleatorios: Densidades conjuntas y densidades marginales. Esperanza y covarianza de un vector aleatorio. Densidades condicionales. Esperanza condicional. Densidad normal multivariada.
        \item 1.10 Procesos Indicadores.
    \end{itemize}

    \item \textbf{2. Cadenas de Markov}
    \begin{itemize}
        \item 2.1 Introducción: Motivación, notación y ejemplo.
        \item 2.2 Cadenas de Markov de tiempo discreto.
        \item 2.3 Propiedades: Clasificación de estados, distribuciones estacionarias y absorción.
        \item 2.4 Casos particulares: Caminatas aleatorias y procesos de ramificación.
    \end{itemize}

    \item \textbf{3. Procesos de Poisson}
    \begin{itemize}
        \item 3.1 Introducción: Motivación, notación y ejemplo.
        \item 3.2 El proceso de Poisson homogéneo en la recta.
        \item 3.3 El proceso de Poisson no homogéneo.
        \item 3.4 Propiedades: Relación con procesos de conteo y de renovación, tiempos de ocurrencia.
        \item 3.5 Proceso de Poisson marcado y generalizaciones a espacio y volumen.
    \end{itemize}
\end{itemize}

\newpage

\subsection{Semestre 2}
\subsubsection{Modelación Analítica}

En este curso se presentarán métodos matemáticos clásicos para la modelación de distintos fenómenos. En particular se discutirán modelos que se engloban en las ecuaciones en derivadas parciales, complementando así el programa de modelación dinámica del semestre previo. Como tal, el programa guarda estrecha relación con ciencias naturales, ecuaciones en derivadas parciales, análisis, cómputo científico y probabilidad. A diferencia de un curso clásico de ecuaciones diferenciales, el énfasis se hace en la formulación y conexiones entre distintos modelos y no necesariamente en métodos para su solución

\begin{itemize}
    \item Cálculo Discreto
    \begin{itemize}
    	\item Teorema de la Divergencia en Grafos
    \end{itemize}
    \item El Teorema de la Divergencia
    \item Leyes de Conservación
    \begin{itemize}
        \item Teorema de Transporte
        \item Ejemplos de modelación a partir de la ley de conservación
        \begin{itemize}
            \item Leyes Constitutivas y Fenómenos de Reacción
            \item Mecánica Estadística y Continua
        \end{itemize}
        \item Método de las Características
    \end{itemize}
    \item Modelos Estocásticos
    \begin{itemize}
        \item Cadenas de Márkov discretas
        \item Caminatas aleatorias
        \item Martingalas
        \item Semigrupos y generadores
    	\item Procesos a tiempo continuo y la ecuación maestra
    	\item Límite asintótico de la caminata aleatoria
    	\begin{itemize}
        	\item Fórmula de Stirling
    	\end{itemize}
    	\item Movimiento Browniano
   	 	\begin{itemize}
       		\item Fórmula de Ito
    	\end{itemize}
    \end{itemize}
    \item Métodos Espectrales
    \begin{itemize}
        \item Series de Fourier
        \item Funciones de Bessel
    \end{itemize}
    \item Optimización
    \begin{itemize}
        \item Ecuación de Euler-Lagrange
        \item Cambios de variables
        \item Multiplicadores de Lagrange
    \end{itemize}
    \item Estructuras Analíticas
    \begin{itemize}
        \item Espacios métricos y topología general
        \item Espacios Medidos
        \item Espacios de Integrabilidad: \( L^p \)
        \item Espacios de Hölder: \( C^{k,\alpha} \)
        \item Compacidad
    \end{itemize}
    \item Teoremas de Punto Fijo de Banach y sus aplicaciones
    \item Compacidad secuencial
    \item Teoremas de Bolzano y Heine-Borel
    \item Teorema de Weierstrass
    \item Teorema de Arzelà-Ascoli
\end{itemize}

\subsubsection{Modelos Estadísticos}

Al finalizar el curso los alumnos:
\begin{itemize}
	\item Habrán adquirido una cultura general sobre el razonamiento inductivo deductivo, sobre conceptos básicos relevantes y enfoques recientes en la Teoría de Inferencia Estadística para modelar fenómenos aleatorios repetibles de interés con actitud crítica.
	\item Desarrollarán habilidades computacionales para fortalecer el conocimiento aprendido en la mayoría de los temas tratados en el temario. Se espera que el alumno llegue a manejar de manera fluida el lenguaje gratuito R o en Python.
	\item Desarrollarán habilidades de redacción científica para explicar eficientemente los temas de trabajo asignados y que aprenda a realizar presentaciones concisas, claras, relevantes y bien estructuradas
\end{itemize}

\textbf{Temas y subtemas.}
\begin{itemize}
    \item \textbf{1. Repaso de conceptos básicos de probabilidad relevantes para este curso.}
    \begin{itemize}
        \item Transformaciones de variables aleatorias y de parámetros de una distribución.
        \item La función de distribución empírica y el Teorema de Glivenko-Cantelli.
        \item Cuantiles de una distribución.
        \item El Teorema de la Transformada Integral de Probabilidad y su importancia para simular variables y validar modelos estadísticos.
    \end{itemize}

    \item \textbf{2. Introducción a la modelación estadística.}
    \begin{itemize}
        \item Planteo, estimación, validación, comparación y selección de modelos estadísticos.
        \item Familias importantes de distribuciones y relaciones relevantes entre ellas.
    \end{itemize}

    \item \textbf{3. Conceptos fundamentales de estimación.}
    \begin{itemize}
        \item Estadísticas suficientes y su importancia.
        \item Estimación puntual, por intervalo y por región de los parámetros de un modelo estadístico.
        \item El método de momentos.
        \item La función de verosimilitud.
        \item Cálculo de probabilidades de cobertura de intervalos y regiones de estimación a través de cantidades pivotales o simulaciones.
        \item El Bootstrap para estimar parámetros puntualmente y por intervalo.
        \item Propiedades asintóticas de los métodos de estimación presentados.
    \end{itemize}

    \item \textbf{4. Estimación de parámetros de interés por separado en modelos estadísticos multiparamétricos.}
    \begin{itemize}
        \item La función de verosimilitud perfil.
    \end{itemize}

    \item \textbf{5. Introducción a pruebas de significancia.}
    \begin{itemize}
        \item Enfatizando su uso para validar modelos estadísticos como partición y contraste con las pruebas de modelos propuestas por Neyman y Pearson.
        \item Comparación y selección de modelos estadísticos paramétricos.
        \item El criterio de información de Akaike y discusión sobre su relevancia para comparar y seleccionar modelos estadísticos óptimos.
    \end{itemize}

    \item \textbf{6. El Modelo de Regresión, supuestos, aplicaciones y diagnóstico.}
    \begin{itemize}
        \item El caso lineal y no lineal.
    \end{itemize}

    \item \textbf{7. Descripción breve de modelos estadísticos para situaciones más complejas.}
    \begin{itemize}
        \item (Esta semana podrán elegirse para las presentaciones finales).
        \item Algunos ejemplos de temas: modelos mixtos, modelos jerárquicos, enfoque Bayesiano para inferencia estadística con modelos jerárquicos, situaciones con información ausente como mezclas de distribuciones y el algoritmo EM; inferencia para situaciones de confiabilidad, aplicaciones del Bootstrap e inferencia para Teoría de Colas entre otros.
    \end{itemize}
\end{itemize}
\subsubsection{Optimización}

\begin{itemize}
\item Conocer las condiciones de optimalidad de primer y segundo orden de la Optimización con Restricciones.

\item Conocer e implementar métodos de optimización numérica para la optimización con restricciones que permita crear las bases necesarias para su aplicación en problemas de investigación

\item Conocer e implementar algoritmos de programación lineal: el método simplex, métodos de punto interior y métodos primal-dual.

\item Conocer e implementar algoritmos de programación cuadrática: método deconjuntos activos y proyección de gradiente.

\item Conocer métodos alternativos para el manejo de restricciones: Métodos de Penalización y Lagrangiano aumentado Conocer el método de eliminación de variables.

\item Conocer e implementar el método de programación cuadrática secuencial para problemas no lineales.

\item Conocer los métodos de punto interior para programación no lineal.

\end{itemize}

\\

\textbf{Temas y subtemas.}
\begin{itemize}
    \item \textbf{1. Introducción}
    \begin{itemize}
        \item[a)] Formulación Matemática
        \item[b)] Ejemplo: Un problema de transporte
        \item[c)] Tipos de problemas de Optimización
        \item[d)] Algoritmos de optimización
        \item[e)] Convexidad
    \end{itemize}

    \item \textbf{2. Fundamentos de Optimización sin restricciones}
    \begin{itemize}
        \item[a)] ¿Qué es una solución?
        \item[b)] Algoritmos (una visión preliminar)
        \begin{itemize}
            \item[I.] Búsqueda en línea
            \item[II.] Métodos de región de confianza
        \end{itemize}
    \end{itemize}

    \item \textbf{3. Métodos de Búsqueda en Línea}
    \begin{itemize}
        \item[a)] Tamaño de paso
        \item[b)] Algoritmos para selección del tamaño de paso
    \end{itemize}

    \item \textbf{4. Método de Región de Confianza}
    \begin{itemize}
        \item[a)] Punto de Cauchy
        \item[b)] Método Dogleg
    \end{itemize}

    \item \textbf{5. Métodos de Gradiente Conjugado}
    \begin{itemize}
        \item[a)] Método de Gradiente Conjugado lineal
        \item[b)] Gradiente Conjugado No Lineal
    \end{itemize}

    \item \textbf{6. Introducción al Cálculo Variacional}
    \begin{itemize}
        \item[a)] Problema sin restricciones
    \end{itemize}

    \item \textbf{7. Cálculo Numérico de Derivadas}
    \begin{itemize}
        \item[a)] Aproximación por diferencias finitas
    \end{itemize}

    \item \textbf{8. Métodos de Newton Prácticos}
    \begin{itemize}
        \item[a)] Newton con pasos inexactos
        \item[b)] Métodos de Newton con búsqueda en línea
        \item[c)] Técnicas de región de confianza y modificación del Hessiano
        \item[d)] Métodos de Newton de Región de Confianza
    \end{itemize}

    \item \textbf{9. Métodos Quasi-Newton}
    \begin{itemize}
        \item[a)] El método BFGS
    \end{itemize}

    \item \textbf{10. Mínimos Cuadrados No Lineales}
    \begin{itemize}
        \item[a)] Método Gauss-Newton
        \item[b)] Método Levenberg-Marquardt
    \end{itemize}

    \item \textbf{11. Métodos de penalización para problemas no lineales con restricciones}
    \begin{itemize}
        \item[a)] Penalización cuadrática
    \end{itemize}

    \item \textbf{12. Algoritmos sin derivadas}
    \begin{itemize}
        \item[a)] Método de Nelder-Mead (Simplex)
        \item[b)] Recocido Simulado
        \item[c)] Algoritmos Bio-inspirados
    \end{itemize}
\end{itemize}
\newpage
\subsection{Semestre 3}
\subsubsection{Seminario de Tesis I}
\subsubsection{Optativa I}
\subsubsection{Optativa II}
\subsection{Semestre 4}
\subsubsection{Seminario de Tesis II}
\subsubsection{Optativa III}

\newpage
\section{Optativas de mi interés.}
\subsection{Álgebra lineal numérica.}
En este curso se presentarán:
\begin{itemize}
	\item Métodos para resolver sistemas de ecuaciones lineales,
	\item Problemas de valores propios de una matriz y
	\item Mínimos cuadrados
\end{itemize}
aplicados a encontrar soluciones numéricas de ecuaciones en derivadas parciales. 

En particular, \textbf{es de especial interés estudiar el caso en que las matrices son de grandes dimensiones}, y además, ralas. Esto presenta dificultades especiales, tanto desde el punto de vista computacional, como algorítimico, por lo que\textbf{ es necesario el estudio de algoritmos, tanto directos como iterativos, así como de métodos de descomposición y factorización de matrices.}
\newpage
\subsubsection{Temas y subtemas.}
\begin{itemize}
    \item \textbf{1. Métodos Directos}
    \begin{itemize}
        \item 1.1 Descampsación LU y de Cholesky
        \item 1.2 Estrategias de pívoteo
        
        \item 1.3 Transformaciones de Householder y Givens
        \item 1.4 Factorización QR
    \end{itemize}

    \item \textbf{2. Análisis Matricial}
    \begin{itemize}
        \item 2.1 Normas y convergencia
        \item 2.2 Errores de redondeo
        \item 2.3 Descomposición en valores singulares
        \item 2.4 El problema de mínimos cuadrados
    \end{itemize}

    \item \textbf{3. Métodos Iterativos}
    \begin{itemize}
        \item 3.1 Métodos de Jacobi, Gauss-Seidel, y SOR
        \item 3.2 Métodos del gradiente conjugado
        \item 3.3 Análisis de convergencia
        \item 3.4 Precondicionamiento para gradiente conjugado
    \end{itemize}

    \item \textbf{4. Problemas Simétricos}
    \begin{itemize}
        \item 4.1 Reducción a forma tridiagonal
        \item 4.2 El método QR simétrico
        \item 4.3 Polinomios ortogonales
        \item 4.4 Algoritmo de Lanczos
    \end{itemize}

    \item \textbf{5. Manejo de Matrices}
    \begin{itemize}
        \item 5.1 Minimización del almacenamiento
        \item 5.2 Matrices en banda y reordenamiento
        \item 5.3 Métodos de factorización
        \item 5.4 Métodos iterativos
    \end{itemize}

    \item \textbf{6. Cálculo Paralelo}
    \begin{itemize}
        \item 6.1 Estructuras de datos distribuidas
        \item 6.2 Multiproceso con memoria compartida
        \item 6.3 Métodos de factorización
        \item 6.4 Problemas de valores propios
    \end{itemize}
\end{itemize}
\newpage
\subsection{Ecuaciones diferenciales parciales.}

\textbf{El curso es sobre la teoría clásica de EDP.} La ecuación de Laplace se estudia con la profundidad de Gilbarg \& Trudinger [2]. La exposición de las ecuaciones del Calor y Onda sigue Evans [2] y John [3].

\subsubsection{Temas y subtemas.}
\begin{itemize}
    \item \textbf{1. Introducción a las EDP}

    \item \textbf{2. La Ecuación de Laplace}
    \begin{itemize}
        \item 2.1 Las Identidades de Green
        \item 2.2 El principio de Máximo
        \item 2.3 El Problema de Dirichlet, Método de Perron
        \item 2.4 La ecuación de Poisson
    \end{itemize}

    \item \textbf{3. La Ecuación del Calor}
    \begin{itemize}
        \item 3.1 El Problema de Cauchy
        \item 3.2 El principio de Máximo
        \item 3.3 Unicidad y Regularidad
    \end{itemize}

    \item \textbf{4. La Ecuación de Onda}
    \begin{itemize}
        \item 4.1 Eliminación de fuentes exteriores
        \item 4.2 Eliminación de reflexión de Hadamard
        \item 4.3 El principio de Duhamel
    \end{itemize}

    \item \textbf{5. Ecuaciones de primer orden}
    \begin{itemize}
        \item 5.1 La ecuación del transporte
        \item 5.2 Ecuación no lineal. Características
        \item 5.3 Ley de conservación escalar
    \end{itemize}

    \item \textbf{TEMAS COMPLEMENTARIOS}

    \item \textbf{6. Métodos de Energía}
    \begin{itemize}
        \item 6.1 Principio de Dirichlet en el problema de Poisson
        \item 6.2 Unicidad hacia arriba: la ecuación del calor
        \item 6.3 Velocidad de propagación finita en la ecuación de onda
    \end{itemize}

    \item \textbf{7. Transformadas integrales}
    \begin{itemize}
        \item 7.1 Fourier y Laplace
    \end{itemize}

    \item \textbf{8. Series de Potencias}
    \begin{itemize}
        \item 8.1 Teorema de Cauchy-Kovalevskaya
    \end{itemize}
\end{itemize}

\newpage

\subsection{Analísis funcional aplicado.}

Estudiar los siguientes problemas:

\begin{itemize}
    \item[a)]\textbf{ La solución de ecuaciones funcionales:} requiere del lenguaje de operadores (no necesariamente lineales) entre espacios de funciones
    \item[b)]\textbf{ Optimización:} requiere del lenguaje de las funcionales sobre dichos espacios
\end{itemize} 
Así pues, se trata de los problemas:

\begin{itemize}
    \item[a)] \( Ax = b \), o bien \( Fx = 0 \), donde \( A, F: X \to X \) son operadores, con \( A \) lineal, \( b \in X \) y \( F \) en general no-lineal, y
    \item[b)] \( \min x \in B^{\emptyset} \), donde \( B \subset X \) y \( \emptyset :X \to \mathbb{R} \) es un funcional.
\end{itemize}

Los métodos de solución de estos problemas necesitan algún tipo de proceso iterativo.

Además, deben enfatizarse los aspectos constructivos y de aproximación. Por lo tanto, el espacio \( X \) donde se plantean los problemas mencionados deberá tener un mínimo de estructura algebraica y topológica. En la mayoría de los casos, \( X \) sería de Hilbert, para aplicar la rica estructura geométrica de estos espacios, aunque algunas aplicaciones harán necesario considerarlo de Banach y otras de Sobolev. Se supone que el alumno habrá estudiado Análisis Real y Teoría de la Medida e Integración, pero no que ya haya cursado Análisis Funcional.

\newpage

\subsubsection{Temas y subtemas}

\begin{itemize}
    \item \textbf{Espacios de Hilbert y de Banach}
    \begin{itemize}
        \item Espacios normados y completitud
        \item Separabilidad, desarrollos ortogonales y ortogonalización
        \item Espacios de Sobolev
        \item Problemas de norma mínima
    \end{itemize}

    \item \textbf{Operadores y funcionales}
    \begin{itemize}
        \item Funcionales lineales
        \item Ejemplos de espacios duales
        \item Operadores lineales acotados
        \item El operador adjunto y ejemplos
    \end{itemize}

    \item \textbf{Convexidad y optimización}
    \begin{itemize}
        \item Derivadas de Gâteaux y Fréchet
        \item Ecuaciones de Euler-Lagrange
        \item Funcionales y conjuntos convexos
    \end{itemize}

    \item \textbf{Multiplicadores de Lagrange y condiciones de Kuhn-Tucker}

    \item \textbf{Control óptimo}
    \begin{itemize}
        \item Controlabilidad y observabilidad
        \item Problemas de tiempo mínimo
        \item La ecuación matricial de Riccati
        \item Programación dinámica
    \end{itemize}

    \item \textbf{Elemento Finito}
    \begin{itemize}
        \item Interpolación y splines
        \item El método de Ritz-Galerkin
        \item Problemas de valores propios
        \item Problemas con valores iniciales
    \end{itemize}

    \item \textbf{Ecuaciones no-lineales}
    \begin{itemize}
        \item El principio de contracción
        \item El método de Newton
        \item Convergencia a la Kantorovitch
        \item Solución aproximada de ecuaciones funcionales
    \end{itemize}
\end{itemize}
\newpage
\section{Investigadores.}

\end{document}
