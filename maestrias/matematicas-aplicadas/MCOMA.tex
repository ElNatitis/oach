\documentclass[10pt,a4paper]{article}
\usepackage[utf8]{inputenc}
\usepackage[spanish]{babel}
\usepackage{amsmath}
\usepackage{amsfonts}
\usepackage{amssymb}
\usepackage{graphicx}
\usepackage[left=2cm,right=2cm,top=2cm,bottom=2cm]{geometry}

% Para tener letras bonitas
\usepackage{fontspec}
\setmonofont{Chomskys.otf}
\newcommand{\titulaso}[1]{{\fontspec{Chomskys.otf} #1}}

% Datos
\title{\titulaso{Maestría en Ciencias con Orientación en Matemáticas Aplicadas}}
\author{natetas}
\date{}

\begin{document}
\maketitle
\newpage
\tableofcontents
\newpage
\section{Sobre el instituto.}
Los programas de Maestrías en Ciencias orientados a la investigación ofrecen varias áreas de especialidad, todas registradas en el Sistema Nacional de Posgrados del CONACHYT.

En el CIMAT los investigadores son quienes imparten clases como profesores de los programas de posgrado. La gran mayoría de ellos pertenecen al \textbf{Sistema Nacional de Investigadores (SNI).} Junto con los investigadores ordinarios, investigadores visitantes y posdoctorales que recibe el Centro regularmente, aportan a la formación de los estudiantes y extienden su perspectiva científica.

\section{Admisión al Programa.}
La admisión al Programa de Maestría en Ciencias con Orientación en Matemáticas Aplicadas se lleva a cabo anualmente. \footnote{Bajo circunstancias excepcionales, a juicio de los coordinadores respectivos, se considerarán admisiones en fechas distintas a las usuales. }Para ingresar el aspirante deberá cumplir con lo siguiente: 

\begin{enumerate}
	\item Cumplir los requisitos de admisión que indique el Reglamento General de Estudios de Posgrados (RGEP) de CIMAT.
	
	\item Presentar y aprobar el examen de admisión ante un comité de admisión designado por el \textit{Comité Académico del Posgrado (CAP).} En el examen para ingreso a la maestría, se valorará el \textbf{manejo eficiente de nociones básicas de Ecuaciones Diferenciales, Cálculo, Álgebra lineal, Estadística y Programación}, problemas de habilidad matemática, así como la motivación del aspirante hacia los estudios de posgrado en el área elegida.
	
	\item Presentarse a una entrevista de preselección ante un comité de admisión designado por el CAP.
	
	\item En base al desempeño académico destacado del solicitante, y bajo recomendación del comité de admisión, el CAP podrá convalidar la presentación del examen de admisión.  
\end{enumerate}
\newpage
\subsection{Temario del examen.}
\begin{enumerate}
	\item \textbf{Cálculo}
	\begin{enumerate}
		\item \textit{Geometría del espacio euclidiano.}
		\begin{enumerate}
			\item Producto interno.
			\item Vectores en el espacio tridimensional y producto cruz.
			\item Coordenadas esféricas y cilíndricas.
		\end{enumerate}
		\item \textit{Diferenciación.}
		\begin{enumerate}
			\item Límites, continuidad.
			\item Derivadas, derivadas parciales, regla del producto y regla de la cadena.
			\item Aproximación y polinomio de Taylor. Método de Newton.
			\item Problemas de optimización, ecuaciones de punto crítico, multiplicadores de Lagrange, criterio de la segunda derivada.
		\end{enumerate}
		\item \textit{Integración.}
		\begin{enumerate}
			\item Integral, interpretación geométrica y métodos de integración
			\begin{enumerate}
				\item Fracciones parciales
				\item Sustituciones trigonométricas
			\end{enumerate}
			\item Integral de línea, superficie, flujo y volumen. Fórmulas de cambios de variables.	 
			\item Teoremas fundamentales del cálculo
			\begin{enumerate}
				\item Divergencia
				\item Green
				\item Stokes
			\end{enumerate}
		\end{enumerate}
	\end{enumerate}
	\item \textbf{Álgebra lineal}
	\begin{enumerate}
		\item \textit{Ecuaciones Lineales}
		\begin{enumerate}
			\item Matrices y operaciones elementales
			\item Matrices escalón y solución de ecuaciones
			\item Producto de matrices, matrices invertibles
			\item Determinantes, interpretación geométrica y regla de Cramer
		\end{enumerate}
		\item \textit{Espacios vectoriales}
		\begin{enumerate}
			\item Independencia lineal
			\item Bases y dimensión
			\item Subespacio vectorial
		\end{enumerate}
		\item \textit{Transformaciones Lineales}
		\begin{enumerate}
			\item Núcleo y Rango
			\item Subespacios de matrices
			\item Cambio de bases
			\item Valores y vectores propios
		\end{enumerate}
		\item \textit{Ortogonalidad}
		\begin{enumerate}
			\item Proyecciones
			\item Gram-Schmidt
		\end{enumerate}
	\end{enumerate}
	\item \textbf{Ecuaciones diferenciales}
	\begin{enumerate}
		\item \textit{Ecuasiones de priemr orden}
		\begin{enumerate}
			\item Ecuaciones lineales
			\item Separación de variables
			\item Ecuaciones exactas
		\end{enumerate}
		\item \textit{Ecuaciones lineales de segundo orden}
		\begin{enumerate}
			\item Wronskiano e independencia lineal
			\item Reducción de orden
			\item Variación de parámetros
		\end{enumerate}
		\item \textit{Aplicaciones}
		\begin{enumerate}
			\item Problemas de mezclas
			\item Circuitos eléctricos
			\item Vibraciones mecánicas. Oscilador armónico
		\end{enumerate}
	\end{enumerate}
\end{enumerate}
\newpage
\section{Plan de estudios.}
El alumno deberá \textbf{cursar y aprobar un mínimo de 9 asignaturas del Plan de Estudios y 2 Seminarios de Tesis.} La distribución es como sigue:
\begin{enumerate}
	\item \textbf{En el primer semestre} se espera que el estudiante curse las siguientes materias:
	\begin{enumerate}
		\item Modelos Estocásticos
		\item Modelación Dinámica
		\item Métodos Numéricos
	\end{enumerate}	
	\item \textbf{En el segundo semestre} se espera que el estudiante curse:
	\begin{enumerate}
		\item Modelos Estadísticos
		\item Modelación Analítica
		\item Optimización
	\end{enumerate}
	\item \textbf{El tercer semestre} se espera que el estudiante curse:
	\begin{enumerate}
		\item 2 optativas
		\item Seminario de Tesis I.	
	\end{enumerate}
	\item \textbf{El cuarto semestre} se espera que el estudiante curse:
	\begin{enumerate}
		\item 1 optativa
		\item Seminario de Tesis II
	\end{enumerate}	
\end{enumerate}
\subsection{Optativas de mi interés.}
\subsubsection{Álgebra lineal numérica.}
\begin{itemize}
	\item \textbf{Objetivos:} En este curso se presentarán métodos para resolver sistemas de ecuaciones lineales, problemas de valores propios de una matriz, y mínimos cuadrados, aplicados a encontrar soluciones numéricas de ecuaciones en derivadas parciales. En particular, es de especial interés estudiar el caso en que las matrices son de grandes dimensiones, y además, ralas. Esto presenta dificultades especiales, tanto desde el punto de vista computacional, como algorítimico, por lo que es necesario el estudio de algoritmos, tanto directos como iterativos, así como de métodos de descomposición y factorización de matrices.
\end{itemize}




\end{document}
